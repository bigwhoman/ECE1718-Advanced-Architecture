\smalltitle{Question 1}

\begin{enumerate}
    \item 
    \begin{enumerate}
        
  
    \item The key challenge is that previous cache side-channel attacks relied on 
    L1 cache which is practically small to probe and also infeaseble due to each core having its own 
    distinguished L1 cache in an environment like the cloud but the last level cache is 
    shared across all cores but it is slower in terms of access time which make previous approaches ineffective.\\
    The goal is to actually develop an effective side-channel attack across virtual machines on the cloud which relies on the shared last level cache between 
    these virtual machines regardless of special conditions like shared memory or VMM level vulnerabilities.

    \item Acording to the \textbf{CEASER} paper: \begin{quote}
        the group of lines that
map to the same set of the cache and can cause an eviction is
called an Eviction Set
    \end{quote}
    This basically means that eviction set is the collection of memory segments that when brought to the cache, would 
    replace a certain line and throw out whatever is in that line (flush that line of cache)
    \\ 
    Now Let us analyze \textbf{Algorithm 1} : 
    \begin{itemize}
        \item First foreach (foreach \textit{candiate} $\in$ lines ...): 
        This code will probe each candidate to see whether it conflicts with any members of the conflict set and inputs it if there is no conflict. Output of this 
        loop is actually the initial conflict set which are candidates with no conflicts. So this loop wants to actually collect a set of memory lines that dont evict eachother 
        from the cache until there is no more non-conflicting lines.
        
        \item The second loop is basically 
        checking candidates which are in lines other than 
        the conflict set and if the candidate has any conflicts with any of the members of 
        conflict set, then the inner loop is to find out what element in
        the conflict set was actually conflicting with our candidate and inserting that into the eviction set and taking it out of the conflict set.
        \\
        Now in the inner foreach loop we do not process all candidates as we already know there is 
        a conflict between that candidate and one member in conflict set and we know that no too members of conflict set would conflict so this one candidate is enough.

    \end{itemize}


\end{enumerate}
    \item  \begin{itemize}
        \item APLR is the rate at which CEASER remaps lines of a certain set in the cache which means that if our APLR is K, then each K*W accesses to the cache would cause a set to get remapped. 
        So if the APLR is set to 200 in a 16-way cache, it means that every 200*16 access to the cache would cause a set to be remapped. In the paper we could see that for an APLR=100 there is an approximate 1\% 
        overhead so if we use APLR=200, since we would have half as much remapping, then we could say that the overhead would become less. This remapping overhead does not directly translate to slowdown because cache hits can still occur while remapping occurs 
        and also not all remapped lines cause conflicts (This fact is shown in the paper where APLR=100 has 1\% remapping overhead but only 0.74\% actual slowdown). Like for example in some workloads like a streaming service, evictions are less likely to impact performance so remapping also would not introduce any significant penalties as opposed to 
        databases which this remapping could have a costly performance impact.

        \item So for a 1MB bank of LLC with 16-way associativity, the linesize is 64 bytes which makes the total number of lines 1MB/64B = 16384
        which also means the total number of sets is 16384/16 =1024. According to their analysis you need F=0.42 of lines for an attack which means that L = F * N = 0.42 * 16384 = 6881 lines. 
        Now you need $\frac{R * L^{2}}{2}$ accesses for this attack which the paper presumes R=2, so the total accesses needed would be $\frac{2 * 6881^{2}}{2} = 47.3 * 10^{6}$ accesses. 
        Now the listed cache latency is 24 cycles and their config is at 3.2GHz which makes 7.5ns for each cache access. This means that 
        the total amount of time needed is $47.3*10^{6}*7.5*10^{-9}=0.35$ which is almost the same as their number in the table(0.4)

        \item \begin{itemize}
            \item tCAS : Column Address Strobe latency which is the time between sending a column address and 
            receiving the data

            \item tRCD : Row to Coumn Delay which is the time needed between activating a row and accessing columns in that row
            \item tRP : Row Precharge which is the time needed to precharge a row before another row would be activated
            \item tRAS : Row Access Strobe which is the minimum a row must remain open for a read or write to complete
        \end{itemize}

        \item In the hinted paper there is a phenomenon discussed where different mapping functions can potentially 
        reduce conflicts between cache lines so there could be a case where a remapping could create a better mapping than the original one.
        Now in the hinted paper, we see that their design helps avoid having neighbor lines conflict for same cache locations which is basically the same thing CEASER does 
        but with a different method. So although CEASER would decrease the perfomance a little bit in average case but there would be cases that 
        it unintentionally would create better mappings for better cache utilization and fewer conflicts for certain workloads.
    \end{itemize}
\end{enumerate}