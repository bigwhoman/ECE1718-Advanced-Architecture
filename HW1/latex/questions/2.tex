\smalltitle{Question 2}
\begin{enumerate}
    \item
    First lets see the wanted MIPS code : 
\begin{lstlisting}[style=mystyle] 
.data
    x:	.word	12
    array1_size: .word 16
    array1: .space 160
    array2: .space 256 * 512
    y:      .byte 0
    ourx:   .space 200
    i:      .word 0

.text
.globl main

main:
    la $t1, ourx
    lw $t2, i
    add $t1, $t1, $t2
    lw $t2, ($t1)
    sw $t2, x
victim:
    lw $a0, x
    lw $a1, array1_size
    ble $a1, $a0, reset
    ...
reset:
    lw $t0, i
    addi $t0, $t0, 1
    sw $t0, i
    beq $a0, $a0, main
EXIT:
    jr $ra
\end{lstlisting}

\item We know that bits 11-2 would determin what index 
in our predictor would be used to see that last time this branch was 
taken or not. So $bit[(PC \& 0xFFC)>>2]$ actually determins what instructions 
would be fetched next when a branch instruction is seen. So if we have our first x 
as something less than 16 (like 2) then the branch would be taken(executes the victim code) and since 
we have a \textbf{last-outcome predictor} and the previous branch was taken, predictor would assume we 
will also take it this time and if we choose our second x as something out of bound(256), then $array1[256]$ could potentially be fetched and cached. 
So $ourx=[2, 256]$

\item 
So with a biomodal predictor, we actually use 2 bits to determin wether a branch has 
been taken or not. If we assume initially all entries are initialized to weakly taken, then 10 would 
be the initial value of the branch predictor for that branch. 
Now, if our next x is 256, then it would predict taken because it has the initial state weakly taken (although it is not taken) and would fetch array1[256].
So $ourx=[256]$

\item The global history register has 4 entries which means we could have 16 states for the GHT (NNNN - NNNT - ... - TTTT) which look at the previous 4 
branches. Now we assume that we are in a state of NNNN at first and the predictor for every state of the GHT of every branch is weakly not taken(01). To make this 
analysis easier, we name the \textbf{while loop} branch A and \textbf{if statement} branch B
 Now first we enter A and take the branch which transitions our state into NNNT and A[NNNN] = 10(weakly taken). Then we give x=2 as an input to B which would cause 
 a transition in GHT to NNTT and B[NNNT] = 10(weakly taken). Afterwards we go back to A and take it which transitions GHT to NTTT and A[NNTT]=10(weakly taken) and after that we give 
 x again as 2 and this would cause to take the branch and GHT becomes TTTT and B[NTTT]=10(weakly taken) and we loop back to A but since this time we see TTTT, we just 
 change A[TTTT]=10(weakly taken) and then we input 2 as x again so that B[TTTT]=10(weakly taken). Now A would be taken again and A[TTTT]=11(strongly taken) and now we know that B[TTTT]=10(weakly taken) so the 
 branch would be predicted as taken and we could input 256 as x and access the out-of-bound data.
 So $ourx=[2, 2, 2, 256]$\\ 
 \textbf{**Note**:} Of course the first state of the GHT and A[GHT] and B[GHT] would determin ourx but the worst case 
 is that we start in NNNN and every entry for B[GHT] and A[GHT] is 00(strongly not taken), in this case if we use $ourx=[2,2,2,2,256]$ then in the first 2 
 itterations GHT would turn from NNNN to TTTT (A and B would be taken 2 times each) and then in the third itteration, B[TTTT]=01(weakly not taken) and in the 4th itterations B[TTTT]=10(weakly taken)
 and in the fifth itteration it would be taken so we input 256. 
\end{enumerate}